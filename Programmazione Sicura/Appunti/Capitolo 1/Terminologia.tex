\chapter{Terminologia}

\section{Asset}
Un \textcolor{red}{asset} è un'entità generica che interagisce con il mondo circostante. Può essere un edificio, un computer, un algoritmo, una persona.
Nell'ambito di questo corso l'asset è un \textcolor{green}{Software}.
Una persona può interagire con un asset in tre modi:
\begin{itemize}
    \item correttamente
    \item non correttamente, in modo involontario
    \item non correttamente, in modo volontario/malizioso
\end{itemize}
Un uso non corretto di un asset può portare a gravi danni come il furto, la modifica o distruzione di dati sensibili, la compromissione di servizi.

\section{Minaccia}
Una \textcolor{red}{minaccia} è una potenziale causa di incidente, che comporta un danno all'asset.
Le minacce possono essere:
\begin{itemize}
    \item accidentali
    \item dolose
\end{itemize}
Microsoft classica le minacce con l'acronimo STRIDE:
\begin{itemize}
    \item Spoofing
    \item Tampering
    \item Repudiation
    \item Information Disclosure
    \item Denial of Service
    \item Elevation of Privilege
\end{itemize}

\section{Attacante}
Un \textcolor{red}{attacante} tenta di interagire in modo malizioso con un asset con lo scopo di tramutare una minaccia in realtà.
Talvolta un attaccante interagisce in modo non malizioso per stimare i livelli di sicurezza.
Distinguiamo tre tipi di attacanti:
\begin{itemize}
    \item \textcolor{red}{White Hat}, fini non maliziosi
    \item \textcolor{red}{Black Hat}, fini maliziosi o tornaconto personale
    \item \textcolor{red}{Gray Hat}, viola asset e chiede denaro per sistemare la situazione
\end{itemize}
