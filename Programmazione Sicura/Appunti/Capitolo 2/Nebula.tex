\chapter{Nebula}
Nebula è la prima macchina virtuale che studieremo in questo corso. Ci sono diversi livelli, noi affronteremo le sfide:
\begin{itemize}
    \item Nebula 00
    \item Nebula 01
    \item Nebula 02
    \item Nebula 04
    \item Nebula 07
    \item Nebula 10
    \item Nebula 13
\end{itemize}
La macchina virtuale è scaricabile dal sito \href{https://exploit.education/}{Exploit Education}.
Le sfide di nebula trattano l'iniezione locale e remota di codice.

Ogni macchina ha tre account:
\begin{itemize}
    \item \textcolor{red}{Giocatore}, un utente con il ruolo di attaccante che può accedere con la coppia di credenziali:
    \begin{itemize}
        \item username: levelN(N=00,01,02,ecc.)
        \item password: levelN
    \end{itemize}
    \item \textcolor{red}{vittima}, chiamati flagN(N=00,01,ecc.) rappresentano la vittima e presentano diversi tipi di vulnerabilità
    \item \textcolor{red}{Admin}, amministratore del sistema con credenziali:
    \item \begin{itemize}
        \item username: nebula
        \item password: nebula
    \end{itemize}
\end{itemize}

Noi accederemo sempre come utente levelN, con l'obiettivo di:
\begin{itemize}
    \item Elevare i privilegi
    \item Ottenere informazioni sensibili
\end{itemize}

Raggiunto l'obiettivo, si cattura la bandierina, per questo motivo le sfide prendono il nome di CTF.

\section{Level00}
Dalla pagina ufficiale si legge:
This level requires you to find a Set User ID program that will run as the “flag00” account.

Quindi dobbiamo trovare un programma con il SETUID acceso che sarà eseguito come se fossimo flag00.

Il primo passo e cercare nella directory principale tutti i programmi con il bit SETUID accesso. Usiamo quindi il comando \text

