\usepackage{graphicx} % Required for inserting images

\usepackage[sfdefault]{roboto}  %% Option 'sfdefault' only if the base font of the document is to be sans serif
\usepackage[T1]{fontenc}

\usepackage[italian]{babel}

\usepackage{setspace}
\setstretch{1.25}

\usepackage[top=3cm, bottom=2.5cm, right=2.75cm, left=2.75cm]{geometry}

\usepackage{amsmath}

\usepackage{hyperref}

\usepackage{xcolor}

\usepackage{listings}
% Configurazione per i comandi di terminale
\lstdefinestyle{bashstyle}{
    language=bash,
    basicstyle=\ttfamily\color{red}, % Imposta il colore del testo a rosso
    keywordstyle=\color{red},
    commentstyle=\color{green},
    stringstyle=\color{red},
    showstringspaces=false,
    breaklines=true,
    columns = flexible,
    frame=none, % Rimuovi il riquadro
}
% Configurazione per il codice Python
\lstdefinestyle{pythonstyle}{
    language=Python,
    basicstyle=\ttfamily\color{black}, % Imposta il colore del testo a nero
    keywordstyle=\color{blue},
    commentstyle=\color{green},
    stringstyle=\color{red},
    showstringspaces=false,
    breaklines=true,
    columns = flexible,
    frame=none, % Rimuovi il riquadro
}
% Configurazione per il codice C
\lstdefinestyle{cstyle}{
    language=C,
    basicstyle=\ttfamily\color{black}, % Imposta il colore del testo a nero
    keywordstyle=\color{orange},
    commentstyle=\color{green},
    stringstyle=\color{blue},
    showstringspaces=false,
    breaklines=true,
    columns = flexible,
    frame=none, % Rimuovi il riquadro
}
% Configurazione per il codice Perl
\lstdefinestyle{perlstyle}{
    language=Perl,
    basicstyle=\ttfamily\color{black}, % Imposta il colore del testo a nero
    keywordstyle=\color{orange},
    commentstyle=\color{green},
    stringstyle=\color{blue},
    showstringspaces=false,
    breaklines=true,
    columns = flexible,
    frame=none, % Rimuovi il riquadro
}